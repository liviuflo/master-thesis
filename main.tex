\documentclass[
	%parspace, % Add vertical space between paragraphs
	%noindent, % No indentation of first lines in each paragraph
	%nohyp, % No hyphenation of words
	%twoside, % Double sided format
	%draft, % Quicker draft compilation without rendering images
	%final, % Set final to hide todos
]{elteikthesis}[2024/04/26]

\pdfobjcompresslevel=0 % Disable object compression

% The minted package is also supported for source highlighting
% See elteikthesis_minted.tex for example
%\usepackage[newfloat]{minted}

% Document's metadata
\title{LiDAR Odometry and Mapping\\Beyond RTK Accuracy } % title
\date{2025} % year of defense

% Author's metadata
\author{Liviu-Daniel Florescu}
\degreetitle{MSc Intelligent Field Robotic Systems (IFRoS)}

% Superivsor(s)' metadata
\supervisor{Iván Eichhardt} % internal supervisor's name
\affiliation{Assistant Professor} % internal supervisor's affiliation
\extsupervisor{Maximilian Fenkart} % external supervisor's name
\extaffiliation{CTO, Sodex Innovations GmbH} % external supervisor's affiliation

% University's metadata
\university{Eötvös Loránd University} % university's name
\faculty{Faculty of Informatics} % faculty's name
\department{Erasmus Mundus Joint Master in \\Intelligent Field Robotic Systems} % department's name
\city{Budapest} % city
\logo{elte_cimer_szines} % logo

% Add bibliography file
\addbibresource{bibliography.bib}

\makeglossaries

\newglossaryentry{surveying}
{
    name=surveying,
    description={The work of examining and recording the area and features of a piece of land in order to construct a map, plan, or detailed description thereof}
}

\newglossaryentry{pointcloud}
{
    name={point cloud},
    description={A set of point coordinates obtained as the output of a scanning sensor}
}

\newglossaryentry{fieldrob}
{
    name={Field Robotics},
    description={The area of robotics that focuses on robots operating in unstructured outdoor environments for tasks like agriculture or exploration}
}

\newglossaryentry{registration}
{
    name={registration},
    description={The process of aligning multiple point clouds in a single coordinate system such that matching features are as close as possible}
}

\newglossaryentry{odometry}
{
name={odometry},
    description={The process of computing relative displacement of a robot using on-board sensors}
}

\newglossaryentry{sfm}{
    name={Structure from Motion (SfM)},
    description={The task of reconstructing a 3D scene and a sequence of camera poses from a set of images}
}

\newglossaryentry{kdtree}{
    name={k-d tree},
    description={A data structure for organizing entries in a $k$-dimensional space using a tree structure. When $k$ is not very large, it provides logarithmic look-up time.}
}

\newglossaryentry{voxel}{
    name={voxel},
    description={A cuboid-shaped region in 3D space; a 3D cell.}
}

\newglossaryentry{keyframe}{
    name={keyframe},
    description={An image frame or 3D scan that is used for odometry estimation. Because modern sensors usually operate at higher frequencies than needed for most algorithms, it is a reasonable decision to skip frames based on a predefined pattern or condition, without sacrificing accuracy.}
}

\newacronym{lidar}{LiDAR}{Light Detection and Ranging}
\newacronym{slam}{SLAM}{Simultaneous Localization and Mapping}
\newacronym{ins}{INS}{Inertial Navigation System}
\newacronym{gnss}{GNSS}{Global Navigation Satellite System}
\newacronym{rtk}{RTK}{Real-Time Kinematics}
\newacronym{kf}{KF}{Kalman Filter}
\newacronym{ekf}{EKF}{Extended Kalman Filter}
\newacronym{pf}{PF}{Particle Filter}
\newacronym{icp}{ICP}{Iterative Closest Point}
\newacronym{ndt}{NDT}{Normal Distributions Transform}
\newacronym{loam}{LOAM}{LiDAR Odometry and Mapping}
\newacronym{lio}{LIO}{LiDAR-Inertial Odometry}
\newacronym{cnn}{CNN}{Convolutional Neural Network}



% The document
\begin{document}

% Set document language
\documentlang{english}

% List of todos (not in the final document)
%\listoftodos[\todolabel]

% Custom commands

\newcommand{\reffig}[1]{(Fig. \ref{fig:#1})}
\newcommand{\refch}[1]{(Chapter \ref{ch:#1})}



% Title page (mandatory)
\maketitle
% Topic declaration page (mandatory) - can also be attached instead
\includepdf[pages={2,3}]{other/TDF_liviu_florescu.pdf}

% Table of contents (mandatory)
\tableofcontents
\cleardoublepage

% Main content
\chapter{Introduction}
\label{ch:intro}

% - General description of robotics;
% - General description of perception;
% - General description of localization.

Like many scientific terms that we encounter in our daily activities, \emph{robotics} describes a broad collection of technologies and stands at the intersection of key research directions. Motivated by practicality, disciplines that appear completely unrelated find themselves building upon advancements in other fields, diluting boundaries and joining forces to enable otherwise-impossible advancements.

What constitutes a \emph{robot}, then? In its inceptive use by Karel Čapek \cite{roberts_robotics_history_1999} in 1921, the word stemmed from the Slavic \emph{robota}, meaning ``servitude'' or ``forced work'', and referred to a human-like mechanical system working on factory assembly lines. This concept, however, dates from earlier centuries. Around 1495, Leonardo Da Vinci envisioned a mechanical knight controlled by a series of pulleys, that was able to perform simple movements \cite{taddei_leonardo_robots_2008}. Testifying to the industrial revolution and the computational breakthroughs of the last few decades, modern-day humanoid robots can perform acrobatics \cite{bostondynamics_backflips_2023}, interact with humans in constrained scenarios \cite{humanoids_hospital} \cite{humanoid_school} \cite{humanoid_school2} and even replicate human facial expressions \cite{humanoid_facial}. Still, a device ought not necessarily appear human-like in order for it to be labeled as a robot. Autonomous vacuum cleaners, space rovers or crop-monitoring drones fall under the same category. At this stage, a complete taxonomy would have to address dozens of physical (size, shape, mobility, locomotion system, etc.) and non-physical (autonomy, perception abilities, use-case, etc.) characteristics, and none of these would independently convey the meaning that we intuitively associate to the notion of \emph{robot}. Without assessing whether an exhaustive, generic definition is even possible, we can synthesize the above by affirming that a \emph{robot} is an artificial system that performs one or more tasks and is able to evaluate its state or gather information from its environment.

\section{Robotic perception}

This formulation highlights two essential aspects of a robotic agent \reffig{robot-env}: actuation, seen as some form of dynamic physical ability, allowing the agent to enact the desired behavior, and perception, the ability to observe changes in the environment.

% Certainly, this  obfuscates several other key elements, such as control mechanisms, information management or a reasoning model, but it emphasizes the importance of perception as a vital link with the physical world. Additionally, this closely resembles the behavior model of humans, where sensing also plays a critical role.

\begin{figure}[H]
    \centering
    \includegraphics[width=0.7\linewidth]{robot-env.jpg}
    \caption[Perception in the Robot-Environment exchange]{A simplified interpretation of the robotics paradigm. The interaction between an agent and its environment can be seen as a two-way flow: the agent alters the environment through its actions, and uses perception to observe it.}
    \label{fig:robot-env}
\end{figure}

% Just as the human senses span multiple stimulus modalities, so does artificial perception, through the use of different types of sensors. An initial categorization differentiates between proprioceptive and exteroceptive sensors. The first type refers to sensors that provide information about the internal state of the system (virtually independent from the environment), while the second type consists of sensors that collect data about the world. Another division classifies sensors as active (emitting some form of energy in order to obtain a reading) or passive (which simply react to external stimuli).

Sensing modalities have largely different contributions to the perception mechanism. To this extent, the phrase \emph{visual dominance} was introduced by F. Colavita \cite{colavita1974human}, whose study demonstrated that humans focus more on the visual component when presented with an audiovisual stimulus, and following research has strongly confirmed this tendency \cite{Hutmacher2019} \cite{hecht2009sensory}. Unsurprisingly, a similar pattern is emerging in the case of robots, thanks to the reduced cost, familiarity and wide availability of cameras. In many situations, visual stimuli provide most of the necessary information, and this has motivated the development of various image processing algorithms.

Technological innovation in the last century has led to the situation in which artificial sensors surpass humans in both the range of signals that are perceived, as well as the accuracy of the measurements. A relevant example is the class of \acrfull{lidar} sensors which retrieve three-dimensional information about the environment at a very high frequency and with rather negligible measurement errors, in the form of \emph{\glspl{pointcloud}}. Among many applications, this type of sensors can be used to construct virtual representations of a specific environment, enabling engineers to experiment with a realistic model, evaluate construction progress or validate a finished project.

\section{Problem definition}

The current work addresses a common and well-known problem in the area of \gls{fieldrob}, namely \acrfull{slam}, and combines the practicality of an industrial solution with a perception-based approach.

\begin{figure}
    \centering
    \includegraphics[width=0.6\linewidth]{images/sdx-compact.jpg}
    \caption[SDX-Compact]{The SDX-Compact manufactured by Sodex Innovations GmbH. The set of sensors consists of a 3D LiDAR scanner, three RGB cameras and a high-accuracy positioning system. Image source: \href{https://fieldwork.ch/de/produkte/geopositioning/mobile-datenerfassung/sdx-compact}{Fieldwork}}
    \label{fig:sdx-compact}
\end{figure}

SDX-Compact \reffig{sdx-compact} is the main product of Sodex Innovations GmbH, consisting of multiple sensors that collect spatial and visual data. This module can be easily mounted on an arbitrary vehicle in order to expand its perception capabilities and convert it into a \gls{surveying} device. While the vehicle is moving, the LiDAR sensor captures 3D scans of the surroundings, as well as related metadata (timestamps, localization information, signal intensity etc.). Because the rig includes a high-accuracy \acrfull{ins}, the localization and orientation data can be used to join the collected point clouds and create a global 3D map of the traversed space.

Nonetheless, this suffers from two main limitations:

\begin{compactitem}
    \item Reliance on unstable signal: internally, the INS depends on information from a \acrfull{gnss} receiver, which is limited to outdoor spaces and whose availability varies depending on weather conditions and surroundings (e.g. thick vegetation, tall buildings, bridges).
    \item Unsatisfactory accuracy: when merging point cloud data, positioning or localization errors introduce inconsistencies in the final 3D model, which hinders precise planning and construction.
\end{compactitem}

Our work aims to address these limitations by introducing a component that utilizes the information collected by the LiDAR sensor in conjunction with the existing data. Three research questions have been formulated to guide this process:

\begin{compactenum}
    \item What metrics exist for measuring the accuracy of point cloud \gls{registration}? In this context, registration refers to placing a pair of related point sets in a common reference frame.

    \item Can methods that use only visual information achieve higher quality point cloud registration (3D mapping) than merging based on \acrfull{rtk}? Usually, GNSS systems provide meter-level accuracy. In the current scenario, however, the system is corrected using Real-Time Kinematics, such that the expected error is at centimeter-level.

    \item To what extent is LiDAR-based \gls{odometry} an alternative to GNSS localization? Previous research indicates that the spatial information present in 3D point clouds could be used to compute the relative displacement between consecutive scans, resulting in the ability to estimate odometry (an essential component of robotic localization) without dedicated sensors such as wheel encoders or accelerometers.
\end{compactenum}

The main contribution of the project consists of developing an original framework for localization and mapping based on data collected with an industrial sensor rig. The results are more generic than if a particular physical robotic system were involved, and thus are relevant for virtually any robotic application with a similar setup.

The following chapters of this document will cover related research directions and efforts that our work builds upon \refch{review}, a detailed description of the components and algorithms involved in developing the project \refch{methodology}, an evaluation of the method based on its results \refch{results}, as well as a series of conclusions that were drawn from the overall process \refch{conclusion}.

% evaluation to enable future use
% document structure



% research questions
% motivation/justification
% scope, limitations

\cleardoublepage

\chapter{Background and Literature Review}
\label{ch:review}

The goal of this chapter is to present the research context in which our project was conducted, by looking at common methods for each of the main components and highlighting those that inspired the current approach.

\section{Simultaneous Localization and Mapping (SLAM)}

SLAM constitutes a key research area in robotics, because it is a foundational building block for autonomous operation. This problem occurs when the robot does not have prior access to a map of the environment, so it must construct one while keeping track of its current position (\emph{online} SLAM). If the goal is to optimize the entire sequence of poses along the robot path, this is known as the \emph{full} SLAM problem. \cite{thrun}

Usually, the problem is discretized along the time dimension, such that at time $t$ we aim to compute the posterior function $p(x_t, M|z_{1:t}, u_{1:t})$, where $x_t$ is the current state, $M$ is the map, $z_{1:t}$ represents the set of measurements collected so far, and $u_{1:t}$ the control sequence. As each of these variables can have different concrete representations, depending on the task and the available sensors, a broad range of approaches have been proposed.

% KF and EKF
Aulinas et al. \cite{aulinas2008slam} note that the earliest methods rely on probabilistic models derived from the recursive Bayes rule, as such formulations can provide intuitive representations of the various noisy components involved in a robotic system. When employing the \acrfull{kf} \cite{Kalman1960} or its variations, the robot state, measurements and control inputs are modeled as multi-dimensional Gaussians, whose covariances describe the associated uncertainty. The algorithm alternates between \emph{predictions}, when the state is modified based on $u_t$, and \emph{updates}, when the prediction is evaluated against the latest observation $z_t$, and the state hypothesis is updated accordingly. The \acrfull{ekf} maintains this structure but can accommodate non-linearities in the displacement or measurement functions by using local linear approximations. Such methods have been successfully applied for indoor \cite{davisonEKF}, aerial \cite{luo2013uav}, and underwater \cite{palomer2019inspection} robots.

\acrfull{pf}, introduced by Del Moral \cite{del1997nonlinear}, is a probabilistic approach that relies on the Monte Carlo method. Instead of an analytical form, the uncertainty is accounted for by considering a large set of samples (representing potential states) and weighting them based on measurement likelihood. This has a higher computational cost than the standard KF methods, but is not affected by any linearization limitations. Nie et al. \cite{lcpf2020} implemented a LiDAR SLAM algorithm based on PF localization, albeit for 2D mapping.

Another large group of methods is represented by \emph{Visual SLAM}, when cameras are the main (and sometimes only) sensor used. As early as 1980, Moravec \cite{moravec1980obstacle} developed a robot capable of estimating its motion by matching features in images captured at discrete poses, in a stop-and-go fashion. In 2007, SLAM was being performed using a single handheld camera \cite{davison2007monoslam,klein2007parallel}, triggering a separation from the popular, offline, \gls{sfm} techniques. The solution is even more robust when a stereo system is available \cite{mei2011rslam}, as this helps avoid the geometrical limitations of monocular vision.

\section{Point cloud Registration}

Before discussing LiDAR-based approaches in more detail, let us review the task of point cloud registration. Given two sets of points $P, Q$ in arbitrary reference frames, representing (at least partially) the same scene, the goal is to find the transformation \mbox{$\matx{T} \in \SE{3}$} (translation and rotation) that best aligns the points. This can be expressed as an optimization problem:

\begin{equation}
    \underset{\matx{T}}{\operatorname{argmin}}\,J(\matx{T}P, Q)\comma
\end{equation}
where $J$ is a custom cost function. The particular case where known point correspondences are provided has a closed-form solution \cite{arun1987leastsquares}, but the \acrfull{icp} algorithm \cite{besl1992method} removes this constraint by alternating between correspondence generation --- each point in $P$ is paired with its nearest neighbor from $Q$ --- and minimizing the point-pair distances. This approach is by far the most widespread, thanks to its simplicity, computational efficiency (\eg using a \gls{kdtree} for closest-point search) and many available variations \cite{rusinkiewicz2001efficient,huang2021comprehensivesurveypointcloud}.

Chen and Medioni \cite{chen1991pointtoplane} proposed a ``point-to-plane'' minimization, while Segal et al. \cite{segal2009generalized} formulated a generalized cost function by considering the probabilistic model underlying the point sets, and implemented a ``plane-to-plane'' minimization. Other variants modify the point selection strategy \cite{masuda96icp,turk1994zippered}, apply a weight to each pair \cite{godin1994three}, prune the set of the correspondences \cite{pulli1999multiview,bouaziz2013sparse} or employ a different minimization algorithm \cite{blais1995registering}.

A different class of solutions operates by discretizing the space into \glspl{voxel} and estimating a normal distribution using the points in each voxel. Introduced in 2003, the \acrfull{ndt} \cite{biber2003normal} algorithm is the main competitor to ICP, enabling registration to be performed without the need for point-to-point correspondences. Magnusson et al. \cite{magnusson2007scan} extended this work to 3D scans, and performed a thorough comparison between this and ICP \cite{magnusson2009evaluation} for the challenging task of registering point clouds captured in underground mines, where few usable features are present. More recently, the method has been applied for outdoor mapping \cite{shen2024}, and as the basis for a Lidar Odometry and Mapping solution \cite{chen2021ndt}.

Neural Networks can also be used for point cloud registration, by creating correspondences based on features extracted by a descriptor architecture \cite{gojcic2019perfect,deng2018ppfnet}. This is particularly useful when no initial estimate of the transformation between the two point sets is available --- a scenario that the other approaches cannot directly handle.

\section{LiDAR-based Odometry and Mapping}

Despite their high price range, LiDAR sensors have gained significant popularity in the last decade, becoming the go-to solution for autonomous mobility. Unlike cameras, LiDARs are not dependent on ambient lighting, so they can operate in total darkness, and they provide high-frequency, reliable 3D information without the need for additional processing. Basic applications include obstacle detection \cite{asvadi20163d,chen2017lidar}, but the amount of information they provide makes them excellent for localization and mapping purposes.

\begin{figure}
    \centering
    \includegraphics[width=0.6\linewidth]{images/lidar-odometry-tree.pdf}
    \caption[LiDAR Odometry Approaches]{A visualization of LiDAR odometry methods, grouped by approach: LOAM~\cite{zhang2014loam}, KISS-ICP~\cite{vizzo2023ral},  CT-ICP~\cite{dellenbach2021cticp}, MULLS~\cite{pan2021mulls}, Direct LO~\cite{chen2022directlo}, GenZ-ICP~\cite{lee2024genz}, GLIM~\cite{koide2024glim}, LeGO-LOAM~\cite{legoloam2018}, COIN-LIO~\cite{pfreundschuh2024coin}, LIO-SAM~\cite{shan2020lio}, Fast-LIO~\cite{fastlio}, CAE-LO~\cite{caelo2020}, SSL-LO~\cite{nubert2021self}, DMLO~\cite{li2020dmlo}, LO-Net~\cite{li2019net}, NDT-LOAM~\cite{chen2021ndt}, E-LOAM~\cite{guo2022loam}.}
    \label{fig:lidar-odometry-family}
\end{figure}


A key reference is \acrfull{loam}, the work of Zhang and Singh \cite{zhang2014loam}, who used a motorized, 2-axis LiDAR scanner to generate 3D point clouds. To register consecutive scans, they perform feature extraction (sharp edges or planar patches), identify correspondences to the previous scan, and find the optimal transformation using the Levenberg-Marquardt method \cite{levenberg1944method,marquardt}. To account for displacement during the beam sweep, points are reprojected by linear interpolation, assuming constant velocity --- this is known as \emph{motion/distortion compensation}. Mapping takes place in parallel, at a slower rate, and the current scan is registered to the map created so far. At that time, this implementation achieved the best results\footnote{\url{https://www.cvlibs.net/datasets/kitti/eval_odometry.php}} on the KITTI Odometry Benchmark \cite{geiger2012kitti}. V-LOAM \cite{zhang2015visual} improved this solution by introducing a visual odometry component based on a fish-eye camera.

A few years later, LeGO-LOAM \cite{legoloam2018} extended the feature extraction process by performing segmentation on a range image generated from the 3D point cloud. Clusters are filtered by size, in order to discard features belonging to potentially noisy points. Another modification is that the parameters of the transformation between consecutive scans are optimized separately: $t_z, \theta_{roll}, \theta_{pitch}$ are computed using planar feature correspondences, then fixed during the optimization of $t_x, t_y, \theta_{yaw}$, which only uses edge features. The method is compared to LOAM and achieves higher accuracy in outdoor scenarios, while being an order of magnitude faster.

F-LOAM \cite{wang2021f} proposes a two-step motion compensation process. For odometry computation, the constant velocity model is used, but the points are re-corrected using the optimized pose before being registered to the map. Correspondences are weighted based on the ``local smoothness'' of the feature, and the non-linear optimization is solved using the Gauss-Newton method. The map is not updated at every scan, but only when the translational change reaches a predefined threshold. This is a common \gls{keyframe} selection technique inherited from Visual Odometry.

A slightly different approach introduces the use of inertial sensors, leading to \acrfull{lio}. This aims to compensate for the lack of reliable 3D features in specific environments, as accelerometers can provide satisfactory motion estimates for short displacements. FAST-LIO \cite{fastlio} applies such a technique for a drone equipped with a high-frequency solid-state LiDAR, by performing tightly-coupled fusion: instead of using the IMU output to correct the scan registration, it is applied on the features extracted from the point cloud. COIN-LIO \cite{pfreundschuh2024coin} introduces a photometric error component, based on the beam intensity values returned by the LiDAR. A monochrome ``intensity image'' is constructed and filtered such that matching can occur between consecutive frames, and these new correspondences extend the residual vector that is minimized for odometry computation. The approach achieves state-of-the-art performance on a dataset of geometrically-degenerate scenes (ENWIDE \footnote{\url{https://projects.asl.ethz.ch/datasets/enwide}}).

% DMLO: Deep Matching LiDAR Odometry 2020 

In the class of deep-learning techniques, we highlight Deep Matching LiDAR Odometry (DMLO) \cite{li2020dmlo}, which translates the registration problem into a supervised machine learning task. Point clouds are projected into a 2D map using cylinder encoding, with range and intensity values as channels, and a \acrfull{cnn} architecture is trained to predict correspondences from pairs of projections. Training samples are constructed from a subset of the target dataset, and the method proves robust across different LiDAR hardware, but does not surpass LOAM on the KITTI sequences. In comparison to \cite{li2019net}, this solution does not leave the geometric problem to the inner workings of the CNN.

% KISS-ICP

The approach that our work draws most inspiration from is KISS-ICP \cite{vizzo2023ral}, a LiDAR-only odometry framework built around point-to-point ICP. This method proposes a few modifications that cooperate towards a hardware-agnostic solution, with a small number of adjustable parameters. The first contribution is a constant velocity motion estimation, which provides the optimization step with an initial guess. The same motion estimation is used for distortion compensation. Secondly, feature extraction is replaced by two stages of voxel-based down-sampling: the first down-sampled point cloud is used to extend the map, while the even lower-resolution set is registered against the existing map to compute the pose estimate. Perhaps the key component is an adaptive distance threshold for correspondence outlier removal. This threshold is updated based on the deviation between the predicted and optimized pose, acting as a form of uncertainty estimation. Additionally, the optimization problem employs a robust kernel, whose scale parameter is related to the adaptive threshold.

% ?
% \section{Industrial solutions}

\cleardoublepage

\chapter{Methodology}
\label{ch:methodology}

This chapter will address the core contributions of our project.
We present the hardware involved in sensor fusion, describe the data acquisition procedure, then introduce our final solution, by detailing the research and implementation process, as well as design decisions that had to be made along the way.

\section{Hardware}

In this context, hardware refers to the set of sensors used for data collection, determined by the existing industrial setup. The data capturing system is controlled by an Nvidia Jetson board, which operates as a middle-man for time synchronisation between the sensors, and coordinates the various data streams. Even though the solution was designed such that it does not inherently rely on any particular device, the relationship between hardware capabilities, data quality and final output makes it crucial to understand the sensors involved in the process. Beyond the components described below, the setup includes three industrial grade ArkCam Basic+ wide angle cameras which provide a 1920x1080 RGB stream over Ethernet. These are not utilised within this project, but represent a noticeable motivation for future work directions.

\begin{figure}
	\centering
	\includegraphics[width=0.6\linewidth]{images/sdx-compact-on-top-nobg.png}
	\caption[SDX-Compact]{The SDX-Compact manufactured by Sodex Innovations GmbH. The set of sensors consists of a 3D LiDAR scanner, three RGB cameras and a high-accuracy positioning system. Image source: \href{https://www.geo-lanes.com/sodex-innovations/}{GeoLanes}}
	\label{fig:sdx-compact}
\end{figure}

\subsection{LiDAR Sensor}

The SDX-Compact \reffig{sdx-compact} is equipped with a Pandar XT32 \cite{hesai_xt16_32_32m} LiDAR sensor, manufactured by Hesai Technology. This is a mechanical rotating LiDAR with a full $360 \degree$ horizontal field of view and 32 beams distributed vertically, at $1 \degree$ resolution. With our settings, the sensor produces 10 complete scans per second, resulting in a horizontal resolution of $0.18 \degree$. The maximum operational range is 120m, decreasing to 50m for low-reflectivity targets. The official specifications state a typical accuracy of $\pm 1$cm, with precision $\pm 0.5$cm, in a static environment. For each beam, the strongest return is processed, leading to 640,000 points being generated per second. The high output bandwidth is handled by an Ethernet connection, over which points are sent as \acrshort{udp} packets. The sensor also supports \acrshort{ptp} synchronisation, essential for high-quality sensor fusion.

The same device has been used for LiDAR odometry applications in the past \cite{kicp}, and has similar specifications to other popular scanners, such as Velodyne VLP-32C, Ouster OS1-32 or Robosense Helios 32.

\subsection{GNSS/INS receiver}

Another component of the sensor stack is the Septentrio AsteRx SBi3 Pro+ GNSS/INS receiver \cite{Septentrio_AsteRx_SBi3_Pro+}, which provides global positioning and orientation data at a rate of 100Hz. Internally, this relies on two distinct mechanisms.

The localization information comes from a dual antenna GNSS module compatible with several GNSS consellations (\eg \acrshort{gps}, \acrshort{glonass}, Galileo), to ensure optimal worldwide coverage. In standalone mode, the advertised typical accuracy is 1m, but the receiver also acts as an NTRIP (a protocol for differential GPS) client, gathering correction information, in order to achieve centimeter-level accuracy.

An \acrfull{imu} module records acceleration data and provides the remaining orientation angles (roll, pitch, yaw) to compute the complete pose, in 6 \acrfull{dof}. This is integrated with the absolute GNSS measurements using the patented FUSE+ technology \cite{Septentrio_FUSE_Sensor_Fusion}, resulting in an orientation error below  $\text{5-10}\degree$.

Like any system reliant on satellite communication, this will suffer significantly in situations where the signal propagation is disturbed (heavy clouds, ``urban canyons'', thick vegetation, spoofing), even leading to loss of \emph{\gls{gnssfix}}.

To conclude this section, we recognize and underline the importance of accurate extrinsic calibration between the LiDAR and the local INS coordinate frame, which has to be performed prior to any reliable data collection procedure. Given the radically different modalities of these two sensors, this is not a trivial task \cite{lidar-gps-calib} and lies outside the scope of the current work.

\begin{figure}
	\centering
	\subcaptionbox{The trajectory, overlaid on a satellite image of the region.}{
		\includegraphics[width=0.45\linewidth]{images/example-trajectory.png}}
	\hspace{1pt}
	\subcaptionbox{Standard deviation of GNSS readings along the trajectory.}{
		\includegraphics[width=0.45\linewidth]{images/example-trajectory-sigma.png}}
	\caption[Example dataset trajectory]{An example dataset collected in rural Germany (Lienziegen).}
	\label{fig:example-trajectory}
\end{figure}

% 48.97905882 lat
% 8.86602707 lon

\section{Data acquisition and pre-processing}

Data is collected by mounting the sensor rig on the roof of a vehicle, and driving around a target area at a relatively low speed (\eg up to 40km/h). \reffig{example-trajectory} In comparison with other public datasets \cite{nuscenes} \cite{pixset}, capturing data in non-urban environments introduces some challenges --- scenes dominated by vegetation, with few identifiable features, uneven terrain --- while minimizing others --- negligible amount of dynamic objects, no tall buildings that could affect GNSS signal propagation.

The raw sensor output is uploaded to a cloud storage facility, and is later processed into \emph{data frames} that fuse the available information \reffig{data-sync}. This can occur because the sensors are PTP-synchronized on initialization, and all recorded data is timestamped.


\begin{figure}
	\centering
	\includegraphics[width=0.7\linewidth]{images/data_sync.jpg}
	\caption[Data frame synchronisation]{The data synchronisation process.\\Time is discretized into fixed-size intervals, resulting in data frames of custom resolution. These are populated with information from the two sensors: LiDAR points arrive in groups, as UDP packets, while GNSS readings have a frequency of approx. 100Hz. We employ a data frame size of 10ms to ensure that each interval has an associated GNSS measurement, and place all corresponding packets in the same data frame.}
	\label{fig:data-sync}
\end{figure}

The next pre-processing step consists of dividing the sequence of data frames such that we operate on individual scans (also known as sweeps) produced by the LiDAR. A scan corresponds to a complete $360 \degree$ rotation, which takes 100ms, so we join the points in 10 consecutive data frames to obtain a single scan.

Every INS reading undergoes a map projection, to obtain $x,y,z$ coordinates in the East-North-Up frame. We consider the frame of the GNSS receiver as the $ego$ coordinate system. The roll $\phi$, pitch $\theta$ and yaw angles $\psi$ determine the absolute orientation, so we compute a global pose $\egoposet{} \in \SE{3}$ as:

\begin{equation}
	\notag
	\egoposet{} = \begin{bmatrix}
		\matx{R} & \vecx{t} \\ \vecx{0} & 1
	\end{bmatrix} =
	\text{Translation}(x, y, z) \cdot
	\rotmtx{z}{\psi} \cdot
	\rotmtx{x}{\theta} \cdot
	\rotmtx{y}{\phi}
\end{equation}

where $\text{Rot}_k(\alpha)$ is the transformation matrix corresponding to a rotation of $\alpha$ around axis $k$.
Because the sensor provides error estimates in the form of global one-sigma values $\left\{ \sigma_x, \sigma_y, \sigma_z, \sigma_\phi, \sigma_\theta, \sigma_\psi\right\}$, we construct the covariance matrix

\begin{equation}
	\notag
	\matx{\Sigma} = \text{diag}\left(\varx{x}, \varx{y}, \varx{z}, \varx{\phi}, \varx{\theta}, \varx{\psi}\right)
\end{equation}

and transform it using the adjoint map of the rotation component

\begin{equation}
	\notag
	\matx{\Sigma}_{ego} =
	\adjoint{\matx{R}}{\vecx{0}} \cdot
	\matx{\Sigma} \cdot
	\adjoint{\matx{R}}{\vecx{0}}^T
\end{equation}

where
\begin{equation}
	\notag
	\adjoint{\matx{R}}{\vecx{t}} =
	\begin{bmatrix}
		\matx{R}                   & \vecx{0} \\
		\skewsym{\vecx{t}}\matx{R} & \matx{R}
	\end{bmatrix}
	\text{ and }
	\skewsym{\vecx{t}} =
	\begin{bmatrix}
		0    & -t_3 & t_2  \\
		t_3  & 0    & -t_1 \\
		-t_2 & t_1  & 0
	\end{bmatrix}
\end{equation}

A LiDAR range reading $r$, captured at azimuth $\alpha$ with an elevation angle of $\phi$, can be converted to a 3D location in the LiDAR frame:

\begin{equation}
	\notag
	\lidarframe{\vecx{p}}= \begin{bmatrix}
		p_x \\ p_y \\ p_z
	\end{bmatrix}=\begin{bmatrix}
		r \cos{\phi} \sin{\alpha} \\ r \cos{\phi} \cos{\alpha} \\ r \sin{\phi}
	\end{bmatrix}
\end{equation}


If $\lidartoego$ is the pose of the LiDAR in the ego frame (from extrinsic calibration), we can compute the location of a point in the ego frame:

\begin{equation}
	\notag
	\begin{bmatrix}
		{}^{ego}\vecx{p}_\lidartxt \\ 1
	\end{bmatrix}
	= \lidartoego \begin{bmatrix}
		\lidarframe{\vecx{p}} \\ 1
	\end{bmatrix}
\end{equation}


\begin{figure}
	\centering
	\subcaptionbox{Before: an artificial duplicate surface \label{fig:motion-comp-pre}}{
		\includegraphics[width=0.45\linewidth]{images/motion-comp-pre.png}
	}
	\hspace{1pt}
	\subcaptionbox{After: surface is corrected \label{fig:motion-comp-post}}{
		\includegraphics[width=0.45\linewidth]{images/motion-comp-post.png}
	}
	\caption[Motion compensation: before and after]{Motion artifacts and the result of motion compensation. \\Green: points from the beginning of the sweep. Red: points from the end of the sweep. Without motion compensation, the vertical surface yields two conflicting clusters, so the scan cannot be used for accurate mapping in the global frame.}
	\label{fig:motion-comp}
\end{figure}

At this stage, it is worth discussing the distortion effect that occurs when a rotating LiDAR sensor is moved at a relatively high velocity. Because it operates in a relative coordinate frame and different beams of the sweep are fired at different times, the beams corresponding to azimuth $0 \degree$ will fire approximately 100ms earlier than the beams corresponding to azimuth $359 \degree$. Placing the resulting points in the same coordinate frame would lead to undesired artifacts, such as duplicate or warped structures. \reffig{motion-comp-pre}


This open research problem \cite{motion-comp-mcdermott} is commonly addressed by estimating the sensor pose change during a sweep \cite{vicp} \cite{vizzo2023ral}, and IMU integration proves satisfactory \cite{deskewing2020}, given the short time interval involved. In our case, the ego poses
$\left\{\egoposeti{i}{0}, \egoposeti{i}{1}, \dots, \egoposeti{i}{m}\right\}$
computed from INS measurements during sweep $i$ are used to bring all points into the frame defined by $\egoposeti{i}{0}$. A point $^{i,k}\vecx{p}$ belongs to data frame $k$, so it will be replaced by:

\[
	\notag
	\begin{bmatrix}
		^{i,0}\vecx{p} \\ 1
	\end{bmatrix}
	= {}^{i,0}\pose_{i,k} \begin{bmatrix}
		{}^{i,k}\vecx{p} \\ 1
	\end{bmatrix}  = \left({\egoposeti{i}{0}}\right)^{-1} \egoposeti{i}{k} \begin{bmatrix}
		{}^{i,k}\vecx{p} \\ 1
	\end{bmatrix}
\]

We also experimented with approaches that do not rely on the absolute pose measurements for subsections of the sweep, such as interpolation using point timestamps or point indices (based on the order in which the points are returned), but these did not yield better results. A potential drawback of our method is that it disregards the localization noise, which could prove counterproductive if the GNSS receiver has low accuracy.

Through this step we are effectively removing the need for sub-scan information and creating a simpler data structure in which each scan is associated a single $\egoposet{}$ pose (from the first data frame), and the points in a sweep can be treated as a unified set.

Without loss of generality, we transform the sequence of global poses
$\left\{\posei{0}, \posei{1}, \dots, \posei{n}\right\}$
into the frame of the first pose, by multiplying each pose with  $\left(\posei{0}\right)^{-1}$, to obtain $\left\{\pose_0, \pose_1, \dots, \pose_n\right\}$. Naturally, $\pose_0$ will always be $\matx{I}_4$, which simplifies the initialization of the odometry estimation. The covariance of each pose is adjusted  with the adjoint of $\left(\posei{0}\right)^{-1}$.


% TODO write about projection

\section{Solution architecture}

At its core, the odometry and mapping solution that we propose requires minimal input: a sequence of point clouds captured by a moving LiDAR scanner, and a timestamp for each scan. If available, absolute localization/INS measurements can be used as additional input.

Let $\mathfrak{P} = \left\{ P_0, P_1, \dots P_n\right\}$ represent the set of point clouds that we operate on, with point coordinates $P_k = \left\{\vecx{p}_{k,i} \vert \vecx{p}_{k,i} \in \RR^3 \right\}$ expressed in the local frame, $\mathfrak{t} = \left\{ t_0, t_1, \dots t_n\right\}$ the corresponding timestamps, and $\widehat{\mathfrak{T}} = \left\{ \gtposei{0}, \gtposei{1}, \dots \gtposei{n} \right\}$ the ground truth poses at which each scan was captured. In an ideal setting, a point cloud registration algorithm would take $P_i$ and $P_{i+1}$ as input and return the transformation $\Delta\gtposei{i} = \gtposei{i}^{-1} \gtposei{i+1}$, \ie the ground truth displacement between consecutive poses. This would correspond to a perfect odometry model, and the complete trajectory could be reconstructed by accumulating the computed displacement. Unfortunately, with the exception of some simulation environments, such an approach is not actually feasbile. As we have already seen, LiDAR output is not perfect, especially in dynamic scenes, and some scenarios are simply unsuitable for odometry estimation based on 3D features \cite{lidartunnel}.

The output of our solution can be formulated as $\mathfrak{T} = \left\{ \pose_0, \pose_1, \dots \pose_n\right\}$. Pose $\pose_k \in \SE{3}$ represents the estimated location and orientation of the \emph{ego} frame from which the points $P_k$ were observed. The complete \emph{map} estimate, represented as a larger point cloud in global coordinates, is the set of all points, each transformed according to their respective pose:
\[
	M\left(\mathfrak{T}, \mathfrak{P}\right) =
	\left\{
	\vecx{p}_{k,i}^* \Big|
	\begin{bmatrix}
		\vecx{p}_{k,i}^* \\ 1
	\end{bmatrix} =
	\pose_k \begin{bmatrix}
		\vecx{p}_{k,i} \\ 1
	\end{bmatrix}, \vecx{p}_{k,i} \in P_k, k \in \overline{0 \dots n}
	\right\}
\]

This uncovers two possible evaluation directions, which also constitute the high-level aims of our method. On one hand, the output trajectory  $\mathfrak{T}$ should match the actual path of the system as accurately as possible. On the other, the final map $M\left(\mathfrak{T}, \mathfrak{P}\right)$ should not hide or damage small details in the scene, to obtain a high-quality 3D reconstruction. Although not immediately obvious, these can result in conflicting requirements, but they anticipate an underlying optimization problem.

\begin{figure}[h]
	\centering
	\includegraphics[width=0.6\linewidth]{images/kiss-icp-architecture.jpg}
	\caption[KISS ICP Architecture]{The KISS ICP pipeline.\\An incoming scan is motion-compensated using the displacement between the previous estimated poses, then it is downsampled and registered against the local map. The pose estimated by the ICP algorithm is added to the trajectory, and the adaptive threshold is updated by evaluating the accuracy of the motion estimation. This determines the ICP outlier removal threshold, improving the system's ability to react to unexpected motions.}
	\label{fig:kiss-icp-architecture}
\end{figure}

The experimental process began with a simple framework, immitating the KISS-ICP \cite{vizzo2023ral} implementation. \reffig{kiss-icp-architecture} In this approach, LiDAR scans are processed sequentially, attempting to compute the optimal pose at each step, and the map is built incrementally. This is computationally efficient and becomes essential when a robot operates in an unknown environment, as it provides up-to-date localization estimates.

\subsection{Motion prediction}

The first component that we analyse addresses the problem of predicting the next pose in the trajectory. Given the trajectory computed so far $\left\{\pose_0, \pose_1, \dots \pose_t \right\}$, we are looking for $\pose_{t+1}'$ that estimates the location and orientation of scan $P_{t+1}$ in the world frame.

This influences two other elements of the framework: motion compensation (scan de-skewing) and point cloud registration. In the absence of IMU data, the displacement between consecutive scans can be interpolated to correct individual points. The registration uses this predicted pose as an initial guess, for faster convergence, as we will see in a later section.

If the scans are captured at small, equal intervals, as is usually the case for LiDAR sensors, the movement can be approximated by a constant velocity model. The displacement between the previous two scans is propagated using
$\Delta \pose_{t}' \approx \Delta \pose_{t-1} = \pose_{t-1}^{-1}\pose_{t}$, which leads to
$\pose_{t+1}' = \pose_{t} \Delta \pose_{t-1}$. In our case, the time steps are not necessarily equal, and we would like to utilize the GPS measurements, when available, so we experimented with a Kalman Filter for computing the location component of $\pose_{t+1}'$. The state $\bm{x}_k \sim \mathcal{N}(\kfx{k}, \kfP{k}) $ is formulated as:
\begin{align*}
	\kfx{k} & = \left[x_k, y_k, z_k, {v_x}_k, {v_y}_k, {v_z}_k\right]^T                                                    \\
	\kfP{k} & = \text{diag}\left(\varx{x_k}, \varx{y_k}, \varx{z_k}, \varx{{v_x}_k}, \varx{{v_y}_k}, \varx{{v_z}_k}\right)
\end{align*}
Over a time step $\Delta t$, this changes according to a linear model:
\newcommand{\Fdt}{\matx{F}(\Delta t)}
\newcommand{\Qdt}{\matx{Q}(\Delta t)}
\begin{align*}
	\kfxpred{k+1} & =  \Fdt \cdot \kfx{k}                    \\
	\kfPpred{k+1} & = \Fdt^T \cdot \kfP{k} \cdot \Fdt + \Qdt
\end{align*}
where
$
	\Fdt =
	\begin{bmatrix}
		\matx{I}_3 & \matx{I}_3 \Delta t \\
		\matx{0}_3 & \matx{I}_3
	\end{bmatrix}
$ and $\Qdt$ is the estimated process noise, which scales with the time interval. In the update step, we compute the next mean and covariance using a measurement $\vecx{z}_k$, by applying a Kalman gain:
\begin{align*}
	\matx{K}  & = \kfPpred{k+1} \matx{H} \left(\matx{H} \kfPpred{k+1} \matx{H}^T + \matx{R}\right)^{-1} \\
	\kfx{k+1} & = \kfxpred{k+1} + \matx{K} \left(\vecx{z}_k - \matx{H}\kfxpred{k} \right)               \\
	\kfP{k+1} & = \left(\matx{I} - \matx{K} \matx{H}\right) \kfPpred{k+1}
\end{align*}

The $\vecx{z}_k$ consists of a location $\left[x_\text{GPS}, y_\text{GPS}, z_\text{GPS}\right]^T$ and velocity values $\frac{1}{\Delta t}\left[x_\text{GPS} - x_k, y_\text{GPS} - y_k, z_\text{GPS} - z_k\right]^T$, assuming linear displacement. The Jacobian with respect to the state vector is $\matx{I}_6$, and the covariance of the measurement noise $\matx{R}$ can be estimated using the covariance of the GNSS reading. If no GNSS reading is associated with the current scan, the pose estimate is retrieved from the predicted position $\kfxpred{k+1}$, and the update step is performed based on the pose returned by the registration procedure.

% proved a sensible way of computing the next location when the time steps are not fixed, but it
This motion prediction method suffers from several disadvantages which outweigh its benefits for our use case. Firstly, it does not address the rotation component of the poses. Some possible workarounds include copying the rotation from the previous pose and letting the registration component correct that (this has been observed to fail when a larger time gap occurs, especially for curved trajectories \reffig{reg-failure-rotation}), or using the rotation returned by the INS. Secondly, it introduces the need for reliable noise estimation, such that the model is reactive to unexpected acceleration, while filtering out measurement noise. Process noise covariance $\matx{Q}$ and the measurement input covariance $\matx{R}$ require approximations which are difficult to balance for a general solution.

% \begin{figure}[h]
% 	\centering
% 	\includegraphics[width=0.5\linewidth]{images/rotation.png}
% 	\caption[Registration Failure Case (Rotation)]{Registration failure case.\\ Incorrectly estimating the rotation between scans (red and blue) can cause the registration to converge to an erroneous local minimum.}
% 	\label{fig:reg-failure-rotation}
% \end{figure}


\begin{figure}
	\centering
	\subcaptionbox{Registration failure case, when the rotation between consecutive scans (red and blue) is estimated incorrectly. \label{fig:reg-failure-rotation}}{
		\includegraphics[width=0.47\linewidth]{images/rotation.png}
	}
	\hspace{1pt}
	\subcaptionbox{Because constant velocity is assumed, the motion guesses require several steps to re-adjust after a time jump. Trajectory built upwards.\label{fig:motion-model-overshoot}}{
		\includegraphics[width=0.47\linewidth]{images/motion-overshoot.png}
	}
	\caption[Motion model challenges]{Challenges of using a Kalman Filter motion model}
	\label{fig:motion-pred-trouble}
\end{figure}


% - need for covariance estimation
% - additional logic for rotations



% mention gaps in the data
% sequential framework - advantages, need for pose at every step
% explain registration equations
% explain map creation
% describe motion model experiments
% describe attempts to include GPS measurements
% - interpolation with fixed alpha
% - combination weighted by covariances
% describe issues - deviation, equal time steps, unable to correct past poses, when GPS is received
% 
% 

%% fix dots in enumerations

% Similarly, $M\left(\widehat{\mathfrak{T}}, \mathfrak{P}\right)$ represents the ground truth map.



% The challenge lies in 

% \begin{compactitem}
% 	\item The trajectory described by this sequence matches the path of the system in the real world as accurately as possible
% 	\item The full point cloud,
% \end{compactitem}

% intermediate challenge = get as good of a displacement estimate, despite the noise in the data



\cleardoublepage

\chapter{Results}
\label{ch:results}

\section{Parameter analysis}

\subsection{Point cloud voxelization}
\begin{table}[h]
    \centering
    \begin{tabular}{|c|c|c|c|c|c|c|c|c|}
        \hline
        \textbf{Voxel} & \textbf{Median}   & \textbf{ATE}    & \textbf{ATE}  & \textbf{Final } & \textbf{Final} & \textbf{Avg.} & \textbf{Avg.}   & \textbf{Avg.}  \\
        \textbf{Size}  & \textbf{Duration} & \textbf{Trans.} & \textbf{Rot.} & \textbf{Error}  & \textbf{Error} & \textbf{RMSE} & \textbf{Corr.}  & \textbf{Corr.} \\
                       &                   & \textbf{}       & \textbf{}     & \textbf{Trans.} & \textbf{Rot.}  & \textbf{}     & \textbf{Trans.} & \textbf{Rot.}  \\
        \hline
        \hline
        0.1            & 0.2984            & 1.0569          & 0.0265        & 2.5337          & 0.0467         & 0.0555        & 0.0080          & 0.0010         \\
        0.2            & 0.1684            & 1.1386          & 0.0283        & 2.7607          & 0.0511         & 0.0868        & 0.0210          & 0.0024         \\
        0.3            & 0.1586            & 1.2803          & 0.0317        & 3.0500          & 0.0542         & 0.1206        & 0.0351          & 0.0040         \\
        0.4            & 0.1589            & 1.4727          & 0.0384        & 3.5078          & 0.0659         & 0.1545        & 0.0519          & 0.0055         \\
        0.5            & 0.1603            & 1.2081          & 0.0324        & 2.9164          & 0.0568         & 0.1876        & 0.0819          & 0.0077         \\
        0.6            & 0.1635            & 1.6171          & 0.0444        & 3.7190          & 0.0677         & 0.2186        & 0.1069          & 0.0116         \\
        \hline
    \end{tabular}
    \caption{Metrics for varying voxel sizes.}
    \label{tab:voxel_metrics}
\end{table}

\begin{figure}[h]
    \centering
    \subcaptionbox{Distribution of point entropies.}{
        \includegraphics[width=0.45\linewidth]{images/voxel-size-entropy.pdf}
    }
    \hspace{1pt}
    \subcaptionbox{Visual comparison. }{
        \includegraphics[width=0.47\linewidth]{images/voxel-size-comp.png}
    }
    \caption[Voxel size effect on map quality]{Voxel size effect on map quality.}
    \label{fig:gicp-corrections}
\end{figure}



% \begin{figure}[h]
%     \centering
%     \subcaptionbox{Translation corrections.}{
%         \includegraphics[width=0.46\linewidth]{images/gicp_corrections_trans.pdf}
%     }
%     \hspace{1pt}
%     \subcaptionbox{Rotation corrections.}{
%         \includegraphics[width=0.45\linewidth]{images/gicp_corrections_rot.pdf}
%     }
%     \caption[]{
%     }
%     \label{fig:gicp-corrections}
% \end{figure}

\subsection{Local mapping}
%% how do the results change, depending on the number of previous scans
% test with 1, 2, 5, 10, 15

\subsection{Registration strategy}
%% how do the results change, depending on whether we use GICP or not


\section{Odometry evaluation}

\section{Mapping evaluation}

\section{KITTI results}

% Things worth mentioning
% - execution time
\cleardoublepage

\chapter{Conclusion}
\label{ch:conclusion}

This work addressed odometry, a fundamental topic of mobile robotics, with emphasis on the area of perception, and was driven by its applicability to an industrial product, the SDX-Compact manufactured by Sodex Innovations GmbH. Range information from a 360-degree LiDAR sensor is fused with localization and orientation measurements provided by an INS, to increase robustness to GPS perturbations and enable 3D mapping quality beyond the accuracy of RTK-corrected readings.

We provided a review of the current state of research around this problem and its many facets. This was followed by a presentation of the hardware and data that we operate on, and a detailed explanation of the experiments conducted around each component of our final solution. Our contribution is an original displacement estimation method which relies on a two-step scan alignment process (robust ICP and Generalized ICP) and a factor graph structure, where registration results are combined with absolute GPS readings. To enforce point cloud registration constraints, we add ``skip connections'' between non-consecutive poses, based on the ICP-estimated displacement. GPS readings introduce unary factors which prevent trajectory drift and remove the need for a loop closure mechanism. The results show that our method is able to handle LiDAR scans captured at irregular time intervals, noisy GPS, or gaps in the sequence of GPS measurements.

We test our solution in multiple scenarios, with varying parameter configurations, and evaluate odometry on the KITTI dataset \cite{geiger2013vision} and on a custom dataset collected with the SDX-Compact. Unlike other odometry or SLAM approaches, we are also interested in the mapping quality --- the complete point cloud captured along the trajectory --- so we evaluate the output 3D map against GPS-only results. We obtain smoother and more fine-detailed maps, thanks to the registration constraints.

Guided by the research questions laid out in the initial stage of the project, the conducted work included multiple experiments which allow us to provide concrete and detailed answers.
\begin{compactenum}
    \item \textit{What metrics exist for measuring the accuracy of point cloud registration?}

    Point cloud registration can be evaluated in several ways, depending on the use case and the desired result. If ground truth displacement is available (\eg from an external sensor system or by designing a scene with custom objects which can be identified and localized precisely), it can be compared to the transformation computed by the registration algorithm.
    In most scenarios, however, registration accuracy is approximated as an error function between the two scans, after they have been transformed to a common reference frame. The metrics are a combination of distance measurements between 3D elements identified in both point clouds, such as points, planes, edges, etc., aiming to quantify the geometric consistency of the combined point cloud. Sensor noise and outliers introduce particular challenges, if a generic metric is sought. In our case, registration accuracy is indicated by the RMSE value obtained after applying GICP, which is a point-to-point metric. For further analysis, we direct the reader to \cite{adolfsson2021coral}.

    \item \textit{Can methods that use only visual information achieve higher quality point cloud registration (3D mapping) than merging based on RTK?}

    \item \textit{To what extent is LiDAR-based odometry an alternative to GNSS localization?}

\end{compactenum}

% discuss general conclusions about the data used and the problems encountered
% discuss research questions

% \section{Future Work}
% Z drift
% integrate RGB data
% improve performance
\cleardoublepage

% \input{samples_en/intro.tex}
% \cleardoublepage

% \input{samples_en/user.tex}
% \cleardoublepage

% \input{samples_en/impl.tex}
% \cleardoublepage

% \input{samples_en/sum.tex}
% \cleardoublepage

% Acknowledgements (optional) - in case your thesis received funding or would like to express special thanks to someone
% \chapter*{\acklabel}
% \addcontentsline{toc}{chapter}{\acklabel}
% In case your thesis received financial support from a project or the university, it is usually required to indicate the proper attribution in the thesis itself. Special thanks can also be expressed towards teachers, fellow students and colleagues who helped you in the process of creating your thesis.

% Appendices (optional) - useful for detailed information in long tables, many and/or large figures, etc.
% \appendix
% \input{samples_en/sim.tex}
% \cleardoublepage


% Bibliography (mandatory)
\phantomsection
\addcontentsline{toc}{chapter}{\biblabel}
\printbibliography[title=\biblabel]
\cleardoublepage

% List of figures (optional) - useful over 3-5 figures
\phantomsection
\addcontentsline{toc}{chapter}{\lstfigurelabel}
\listoffigures
\cleardoublepage

% List of tables (optional) - useful over 3-5 tables
% \phantomsection
% \addcontentsline{toc}{chapter}{\lsttablelabel}
% \listoftables
% \cleardoublepage

% List of algorithms (optional) - useful over 3-5 algorithms
% \phantomsection
% \addcontentsline{toc}{chapter}{\lstalgorithmlabel}
% \listofalgorithms
% \cleardoublepage

% List of codes (optional) - useful over 3-5 code samples
% \phantomsection
% \addcontentsline{toc}{chapter}{\lstcodelabel}
% \lstlistoflistings
% \cleardoublepage

% List of symbols (optional)
%\printnomenclature

{\setlength{\baselineskip}{0.6\baselineskip} % Adjust 0.9 to reduce spacing
    \printglossary[type=\acronymtype]
    \printglossary
}

\end{document}
