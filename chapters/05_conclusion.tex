\chapter{Conclusion}
\label{ch:conclusion}

This work addressed odometry, a fundamental topic of mobile robotics, with emphasis on the area of perception, and was driven by its applicability to an industrial product, the SDX-Compact manufactured by Sodex Innovations GmbH. Range information from a 360-degree LiDAR sensor is fused with localization and orientation measurements provided by an INS, to increase robustness to GPS perturbations and enable 3D mapping quality beyond the accuracy of RTK-corrected readings.

We provided a review of the current state of research around this problem and its many facets. This was followed by a presentation of the hardware and data that we operate on, and a detailed explanation of the experiments conducted around each component of our final solution. Our contribution is an original displacement estimation method which relies on a two-step scan alignment process (robust ICP and Generalized ICP) and a factor graph structure, for integrating GPS readings. To enforce point cloud registration constraints, we add ``skip connections'' between non-consecutive poses, based on the ICP-estimated displacement.

% discuss results
% discuss general conclusions about the data used and the problems encountered

% \section{Future Work}
% Z drift
% integrate RGB data
% improve performance